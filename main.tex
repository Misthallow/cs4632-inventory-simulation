\documentclass{article}
\usepackage{graphicx}
\usepackage{hyperref}
\usepackage{amsmath}

\title{Discrete-Event Simulation of Stochastic Inventory Control Policies for Retail Supply Chains}

\author{
Jasper Zheng \\
CS 4632 -- Modeling and Simulation \\
Instructor: Christopher Regan \\
\date{\today}
}

\begin{document}

\maketitle
\newpage
\section{Project Overview}

\subsection{Project Title}
Discrete-Event Simulation of Stochastic Inventory Control Policies for Retail Supply Chains

\subsection{Domain}
Retail supply chain and inventory management systems.

\subsection{Problem Statement}
Retailers must manage inventory under uncertain customer demand and variable supplier lead times. Poor inventory policies can lead to excessive holding costs or frequent stock outs, reducing service levels, and increasing operational expenses.  
This simulation aims to evaluate and compare different inventory control policies, such as the $(s,S)$ policy and Economic Order Quantity (EOQ), to determine which strategies minimize total cost while maintaining high service levels under stochastic conditions.

\subsection{Scope}

\textbf{Included in simulation:}
\begin{itemize}
    \item Single retail store and single supplier/warehouse
    \item Stochastic customer demand
    \item Inventory Resupply policies
    \item Random lead times for shipments
    \item Cost tracking (holding, shortage, ordering)
\end{itemize}

\textbf{Excluded from simulation:}
\begin{itemize}
    \item Transportation routing optimization
    \item Pricing or marketing effects
    \item Multiple product types
    \item Detailed financial accounting
\end{itemize}

\section{System Description}
\subsection{System Components}

The simulation consists of the following primary entities:

\begin{itemize}
    \item \textbf{Store:} Maintains inventory, fulfills customer demand, and places replenishment orders.
    \item \textbf{Supplier/Warehouse:} Receives orders and ships inventory after a lead time.
    \item \textbf{Inventory Policy:} Determines when and how much to reorder.
    \item \textbf{Order/Shipment:} Represents replenishment transactions.
    \item \textbf{Event Scheduler:} Manages discrete events such as demand arrivals and shipment arrivals.
\end{itemize}

\subsection{System Dynamics}

The system evolves using discrete events:

\begin{enumerate}
    \item Customers generate stochastic demand that reduces store inventory.
    \item If inventory is insufficient, a stockout or lost sale occurs.
    \item When inventory falls below a reorder threshold, the store places an order.
    \item After a stochastic lead time, the shipment arrives and inventory increases.
    \item Costs and performance metrics are recorded continuously.
\end{enumerate}

\subsection{Assumptions}

\begin{itemize}
    \item Initial inventory $I_0 = 200$ units
    \item Holding cost $h = 2$ dollars per unit per day
    \item Ordering cost $K = 200$ dollars per order
    \item Shortage cost $p = 50$ dollars per unit
    \item Simulation duration: 365 days
    \item Single product and single supplier
    \item Independent daily demand
\end{itemize}

\section{Implementation Approach}
\subsection{Programming Language}
Python is selected due to its simplicity, strong numerical libraries, and suitability for rapid simulation development.

\subsection{Development Environment}
\begin{itemize}
    \item NumPy for numerical computation
    \item Matplotlib for visualization
    \item Optional: SimPy for event scheduling
    \item GitHub for version control
\end{itemize}

\subsection{Simulation Type}
Discrete-event simulation.

\subsection{Data Collection Plan}

The simulation will record:

\begin{itemize}
    \item Total cost
    \item Holding cost
    \item Ordering cost
    \item Shortage cost
    \item Stockout rate
    \item Service level (fill rate)
    \item Average inventory level
\end{itemize}

These metrics will be used to compare policy performance.

\subsection{Simulation Algorithm}

The simulation uses an event-driven approach. The main logic is:

\begin{verbatim}
Initialize inventory I0 = 200
Schedule first demand event

while time < 365 days:
    process next event
    
    if demand:
        reduce inventory
        record shortage if needed
        
        if I < s:
            order Q = S - I
            schedule shipment after lead time
    
    if shipment arrival:
        increase inventory
    
    update costs
\end{verbatim}

\section{Literature Review}

\subsection{Core Models and Algorithms}

A minimum of five academic or industry sources will be reviewed. For each source, the following will be discussed:

\begin{itemize}
    \item The model or algorithm presented
    \item Its theoretical foundation
    \item How it applies to this simulation
    \item What adaptations will be made
\end{itemize}

\textbf{Planned topics include:}
\begin{itemize}
    \item $(s,S)$ inventory policies
    \item Economic Order Quantity (EOQ)
    \item Stochastic demand modeling
    \item Discrete-event simulation techniques
    \item Supply chain performance metrics
    \end{itemize}

\subsection{Core Models and Algorithms}

\subsubsection{Stochastic Demand Modeling}

Daily customer demand is modeled as a Poisson random variable:

\[
D_t \sim \text{Poisson}(\lambda)
\]

The probability mass function is:

\[
P(D=k) = \frac{\lambda^k e^{-\lambda}}{k!}
\]

where $\lambda$ is the average demand per day.  
For this simulation, we assume:

\[
\lambda = 50 \text{ units/day}
\]

This model captures random independent customer arrivals commonly observed in retail systems.

\subsubsection{$(s,S)$ Inventory Control Policy}

A continuous-review $(s,S)$ policy is implemented. When the current inventory level $I_t$ falls below reorder point $s$, an order is placed to raise inventory to level $S$:

\[
\text{If } I_t < s, \quad Q = S - I_t
\]

where $Q$ is the order quantity.

Simulation parameters:

\[
s = 100, \quad S = 300
\]

\subsubsection{Economic Order Quantity (EOQ)}

For baseline comparison, the classical EOQ model is used:

\[
Q^* = \sqrt{\frac{2DS}{H}}
\]

where:
\begin{itemize}
    \item $D$ = annual demand
    \item $S$ = ordering cost per order
    \item $H$ = holding cost per unit per year
\end{itemize}

Example parameters:

\[
D = 18{,}250, \quad S = 200, \quad H = 5
\]

\subsubsection{Lead Time Modeling}

Supplier lead time is stochastic and modeled as:

\[
L \sim \mathcal{N}(\mu_L, \sigma_L)
\]

with:

\[
\mu_L = 7 \text{ days}, \quad \sigma_L = 2 \text{ days}
\]

This variability introduces uncertainty in replenishment arrival times.

\subsubsection{Cost and Performance Metrics}

Inventory performance is evaluated using:

\[
\text{Holding Cost} = h \cdot I_t
\]
\[
\text{Ordering Cost} = K \cdot N_{orders}
\]
\[
\text{Shortage Cost} = p \cdot \text{backorders}
\]
\[
\text{Total Cost} = C_h + C_o + C_s
\]

Service level (fill rate):

\[
\text{Fill Rate} = 1 - \frac{E[\text{shortages}]}{E[\text{demand}]}
\]

\section{UML Diagrams}

\subsection{Class Diagram}
\begin{figure}[h!]
    \centering
    \includegraphics[width=0.9\textwidth]{docs/class_diagram.png}
    \caption{Class Diagram of Inventory Simulation}
    \label{fig:class_diagram}
\end{figure}

\subsection{Activity Diagram}
\begin{figure}[h!]
    \centering
    \includegraphics[width=0.9\textwidth]{docs/activity_diagram.png}
    \caption{Activity Diagram showing the main simulation flow}
    \label{fig:activity_diagram}
\end{figure}
\section{Repository}
\texttt{https://github.com/Misthallow/cs4632-inventory-simulation/tree/main}

\newpage
\bibliographystyle{IEEEtran} 
\bibliography{references}
\end{document}
